\documentclass[12pt]{article}
\usepackage{blindtext}
\begin{document}
\begin{titlepage}
\begin{center}
	\LARGE\textbf{Mandelbrot's set}
\end{center}
\large
It is known that the Mandelbrot's set is the set of complex numbers $c$ for which a recurrent sequence $z_{n+1}=z_{n}^{2}+c$ does not diverge to infinity when iterated from $z_{0}=0$.
\begin{flushleft}
	
Consider the number $c=\frac{1}{4}$, which belongs to the set. Let's approach exceedingly close to this point from the right (considering $c=0,\!26$, then $c=0,\!2501$, then $c=0,\!250001$, etc.). In other words, let's pay attention to numbers like $c=\frac{1}{4}+\varepsilon$, where $\varepsilon={10}^{-2}, {10}^{-4}, {10}^{-6},\ldots$, figuring out how many iterations it is needed so that these numbers exceed $2$ (this is a so-called "infinity criterion"). It turns out that a needed number of iterations is connected with $\pi$.

\end{flushleft}
\end{titlepage}
\end{document}