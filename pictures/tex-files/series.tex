\documentclass[12pt]{article}
\usepackage{blindtext}
\begin{document}
\begin{titlepage}
\begin{center}
	\LARGE\textbf{Series' summation}
\end{center}
\large
This method of computation of $\pi$ is based on series' summation (and fractions' multiplication). There are many series, whose amounts (and many fractions, whose products) contain $\pi$ in the answer. These are just a few of them:

\begin{flushleft}
	
\textbf{1) Gregory-Leibniz series}:

$\pi=4\cdot\left(1-\frac{1}{3}+\frac{1}{5}-\frac{1}{7}+\frac{1}{9}-\ldots\right)=4\cdot\displaystyle\sum_{i=0}^{\infty} \frac{{\left(-1\right)}^{i}}{2i+1}$.

\textbf{2) Madhava series}:

$\pi=\sqrt{12}\cdot\left(1-\frac{1}{3\;\cdot\;3}+\frac{1}{5\;\cdot\;{3}^{2}}-\frac{1}{7\;\cdot\;{3}^{3}}+\ldots\right)=\sqrt{12}\cdot\displaystyle\sum_{i=0}^{\infty}\frac{{(-1)}^{i}}{\left(2i+1\right)\cdot{3}^{i}}$.

\textbf{3) Nilakantha series}:

$\pi=3\!+\!\frac{4}{2\;\cdot\;3\;\cdot\;4}\!-\!\frac{4}{4\;\cdot\;5\;\cdot\;6}\!+\!\ldots=3\!+\!4\!\cdot
\!\displaystyle\sum_{i=0}^{\infty}\frac{{\left(-1\right)^{i}}}{\left(2i+2\right)\cdot\left(2i+3\right)\cdot\left(2i+4\right)}$.

\textbf{4) Euler series}:

$\frac{{\pi}^{2}}{6}=1+\frac{1}{{2}^{2}}+\frac{1}{{3}^{2}}+\frac{1}{{4}^{2}}+\ldots=\displaystyle\sum_{i=0}^{\infty}\frac{1}{{\left(i+1\right)}^{2}}$.

\textbf{5) Wallis' formula}:

$\frac{\pi}{2}=\frac{2}{1}\cdot\frac{2}{3}\cdot\frac{4}{3}\cdot\frac{4}{5}\cdot\frac{6}{5}\cdot\frac{6}{7}\cdot\ldots=\prod\limits_{i=0}^{\infty}\frac{{\left(2i+2\right)}^{2}}{\left(2i+1\right)\;\cdot\;\left(2i+3\right)}$.

\end{flushleft}
\end{titlepage}
\end{document}