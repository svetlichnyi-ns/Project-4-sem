\documentclass[12pt]{article}
\usepackage{blindtext}
\usepackage{mathtools}
\begin{document}
\begin{titlepage}
\begin{center}
	\LARGE\textbf{Monte Carlo method}
\end{center}
\large
This is the implementation of one of variations of Monte Carlo method for computing $\pi$. Consider a square and a circle inscribed in it. The circle area is $\pi{R}^2$, whereas the square area is $4{R}^2$. Let us randomly scatter points inside a square. The probability for each concrete point to occur within a circle is equal to the ratio of the areas of a circle and a square, i.e. \!\!to $\frac{\pi}{4}$. It allows us to statistically calculate $\pi$, because after scattering $N$ points it turns out that approximately $\frac{\pi}{4}\cdot N$ of them are located inside a circle.

\end{titlepage}
\end{document}
