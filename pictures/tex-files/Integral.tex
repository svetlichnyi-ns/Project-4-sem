\documentclass[12pt]{article}
\usepackage{blindtext}
\begin{document}
\begin{titlepage}
\begin{center}
	\large\textbf{Numerical integration}
\end{center}
\scriptsize
This method of computation of $\pi$ is based on the usage of three integral expressions:
\begin{flushleft}
	$1)\int\limits_{-1}^{1} 2\cdot\sqrt{1-x^{2}} dx=\pi;\;\;$
    $2)\int\limits_{0}^{1} \frac{4}{1+x^{2}} dx = \pi;\;\;$
	$3)\int\limits_{0}^{\frac{1}{2}} \frac{6}{\sqrt{1-x^{2}}} dx = \pi.$

Each of these three integrals may be calculated via the following methods:

\textbf{1) left rectangle method}:

Let the segment of integration $[a; b]$ be divided into $N$ equal parts. Then the integral can be numerically calculated as follows:

$\int\limits_{a}^{b} f(x)dx=\frac{b-a}{N}\cdot\displaystyle\sum_{i=0}^{N-1} f\left(a+i\cdot\frac{b-a}{N}\right)$.

\textbf{2) right rectangle method}:

Under the same assumptions as in the first method:

$\int\limits_{a}^{b} f(x)dx=\frac{b-a}{N}\cdot\displaystyle\sum_{i=0}^{N-1} f\left(a+\left(i+1\right)\cdot\frac{b-a}{N}\right)$.

\textbf{3) middle rectangle method}:

Under the same assumptions as in the first method:

$\int\limits_{a}^{b} f(x)dx=\frac{b-a}{N}\cdot\displaystyle\sum_{i=0}^{N-1} f\left(a+\left(i+\frac{1}{2}\right)\cdot\frac{b-a}{N}\right)$.

\textbf{4) trapezoidal method}:

Under the same assumptions as in the first method:

$\int\limits_{a}^{b} f(x)dx=\frac{1}{2}\cdot\frac{b-a}{N}\cdot\displaystyle\sum_{i=0}^{N-1}\left[ f\left(a+i\cdot\frac{b-a}{N}\right)+f\left(a+\left(i+1\right)\cdot\frac{b-a}{N}\right)\right]$.

\textbf{5) parabola method (Simpson's method)}:

Under the same assumptions as in the first method:

$\int\limits_{a}^{b} f(x)dx=\frac{1}{6}\cdot\frac{b-a}{N}\cdot\displaystyle\sum_{i=0}^{N-1}\left[ f\left(a+i\cdot\frac{b-a}{N}\right)+4\cdot f\left(a+\left(i+\frac{1}{2}\right)\cdot\frac{b-a}{N}\right)+f\left(a+\left(i+1\right)\cdot\frac{b-a}{N}\right)\right]$.

\textbf{6) Romberg's method}:

It uses values, calculated earlier by the trapezoidal method, to obtain a more accurate estimate of the integral.

\textbf{7) one-dimensional Monte Carlo method}:

The segment of integration $[a; b]$ is divided into $N$ equal parts, and the function's value is chosen randomly on each small segment. Consequently,

$\int\limits_{a}^{b} f(x)dx=\frac{b-a}{N}\cdot\displaystyle\sum_{i=0}^{N-1} f\left( rand\left[a+i\cdot\frac{b-a}{N};\;a+\left(i+1\right)\cdot\frac{b-a}{N}\right]\right)$,

where the expression in the function's argument means an arbitrary value within a corresponding segment.

\textbf{8) two-dimensional Monte Carlo method}.

Around the graph of a given non-negative function a rectangle is taken into consideration. We start randomly scattering points inside this rectangle. Based on the number of points that fell under the curvilinear trapezoid we statistically compute the integral.

\end{flushleft}
\end{titlepage}
\end{document}