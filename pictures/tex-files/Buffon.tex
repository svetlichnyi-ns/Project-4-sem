\documentclass[12pt]{article}
\usepackage{blindtext}
\begin{document}
\begin{titlepage}
\begin{center}
	\LARGE\textbf{Buffon's needle problem}
\end{center}
\large

The initial formulation of the problem, set by Buffon, was the following:

"Suppose we have a floor made of parallel strips of wood, each the same width, and we drop a needle onto the floor. What is the probability that the needle will lie across a line between two strips?"

It was the earliest problem in geometric probability. Its solution is: $p=\frac{2}{\pi}\cdot\frac{l}{w}$, where $p$ is the sought probability, $l$ is each needle's length, $w$ is the interval between adjacent lines. So, this can be used as a Monte Carlo method for approximation of $\pi$.

\end{titlepage}
\end{document}