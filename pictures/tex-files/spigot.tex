\documentclass[12pt]{article}
\usepackage{blindtext}
\usepackage{mathtools}
\begin{document}
\begin{titlepage}
\begin{center}
	\LARGE\textbf{A Spigot Algorithm For Digits Of ${\boldsymbol\pi}$}
\end{center}
\large
This method is based on the following expression for $\pi$:
\begin{flushleft}

$\pi=2+\frac{1}{3}\cdot\left(2+\frac{2}{5}\cdot\left(2+\frac{3}{7}\cdot\left(\ldots\left(2+\frac{i}{2i+1}\cdot\left(\ldots\right)\right)\right)\right)\right)$.

The idea is that in the number system with a variable base (which is equal to $\frac{i}{2i+1}$ for $i = 1,2,\ldots$) $\pi$ is equal to $2,\!\left(2\right)$.

\end{flushleft}
\end{titlepage}
\end{document}